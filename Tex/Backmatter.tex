\chapter[致谢]{致\quad 谢}\chaptermark{致\quad 谢}% syntax: \chapter[目录]{标题}\chaptermark{页眉}
%\thispagestyle{noheaderstyle}% 如果需要移除当前页的页眉
%\pagestyle{noheaderstyle}% 如果需要移除整章的页眉

时光飞逝,转眼间就到了毕业的时间。三年的研究生生活历历在目,忙碌而充实。很高兴也很荣幸能来到计算所,尤其是网络数据重点实验室读研,在此,向所有帮助、关心、支持过我的人表示由衷的感谢。

首先,感谢我的导师廖华明老师。在我的研究生期间,廖老师经常关心我的科研情况,在我研一还在怀柔的时候,在科研内容上,我对自己要做的事情也不是很清晰,这时候廖老师就经常给我发邮件,指导我的学习,提醒我及时复习,学习前沿知识,梳理要做的工作,制定详细的计划,让我可以按部就班的按照计划不断学习与实践,我非常感谢廖老师。在平时的工作中,廖老师以身作则,为人师表,三年来,她兢兢业业和认真负责的工作态度深深影响着我,成为我学习的榜样。

感谢实验室刘悦、伍大勇、俞晓明、周术夏等老师的关心和帮助。实验室的各位老师为我们提供了良好的学习环境,并且时刻关心我们的情况。虽然老师们都很忙,但我们仍然能够有问题就可以去寻找帮助,无论是做人、做事,都受益匪浅。

感谢徐学可老师对我的帮助,进入实验室之初,在徐学可老师的帮助下我对主题模型有了初步的认识,并且他强悍的代码能力让我深深折服,让我对以后的科研有了极大的兴趣。

感谢王永庆师兄,张静师姐带我一起做实际的工程项目,甲方需求变动大,项目联调问题多,很多情况下都是师兄师姐帮我们解决问题,处理不必要的杂事,万分感谢。同时再次感谢王永庆师兄和曹婍同学对我毕设的耐心指导,给了我很多思路和建议,对流行度预测这个方向有了更深的认识。

感谢牛国成、吕福煜、周楠、陈绍毅等舆情组的师兄对我的帮助和支持,和你们一起工作学习受益良多,同时还要感谢同组的岳新玉,赵子承,秦宇君,李飞,王宪发,杨发,张随远同学,从你们身上看到了拼搏,看到了无惧艰险。同时还要感谢我的舍友肖岩,肖岩是一位才华横溢的同学,他自制力强,对未来充满信心,并且十分热爱写代码,对科研也有极大的兴趣,和他做舍友十分荣幸。

感谢研一怀柔同宿舍的马晓龙,宋宇,孔德飞,孙明磊,张远,邓果一,郭邯,肖岩,赵子承这些舍友,在怀柔度过了一年快乐的时光,感谢大家的陪伴和支持,怀念那段青葱岁月。

最后,要特别感谢我的父母,父母给了我学习和生活上巨大的鼓励和支持,给我提供了强大的后盾。特别是感谢父母这么多年的无私奉献和陪伴,在我懵懂无知的时候,与我一起渡过难关。

\chapter{作者简历}


\section*{作者简历}

姓名:王新乐 \qquad 性别:男 \qquad 出生日期:1991.09.10 \qquad 籍贯:河北衡水
\\

2015.9 -- 2018.7\qquad 中国科学院计算技术研究所~\qquad ~硕士 

2011.9 -- 2015.7\qquad 北京科技大学~ \qquad \qquad \qquad \qquad 学士 

\section*{攻读硕士期间参加的研究项目}

\begin{enumerate}[label=\text{[}\arabic*\text{]}]
\item 2016年9月--2017年4月 \qquad 苏州移动主题模型项目
\item 2017年7月--2017年9月 \qquad 社群发现项目
\item 2016年7月--2018年3月 \qquad 舆情监控系统研发
\end{enumerate}

\section*{攻读硕士学位期间的获奖情况}

\begin{enumerate}[label=\text{[}\arabic*\text{]}]
\item 中国科学院大学 \qquad \qquad \qquad ~三好学生
\item 中国科学院大学 \qquad \qquad \qquad ~优秀学生干部
\item 中国科学院计算技术研究所 ~~ ~~优秀志愿者
\end{enumerate}
\cleardoublepage[plain]% 让文档总是结束于偶数页,可根据需要设定页眉页脚样式,如 [noheaderstyle]

