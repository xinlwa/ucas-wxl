\chapter{结论和下一步工作}\label{chap:six}

\section{文章总结}

Hashtag 是微博,Twitter 中一种新的主题标签形式,因为使用简单方便,并 且语意鲜明,可以表达用户主观想法,所以吸引了众多用户参与其中,对其流行 度的预测可以来预测某一事件或者某一话题的热度,因此具有很高的研究价值, 但是目前对于主题标签的研究较少。本文的研究目标是对主题标签流行度进行 预测,贡献如下:
\begin{enumerate}
\item \bfseries 基于用户网络结构特征的主题标签流行度预测\mdseries 

从 Hashtag 自身特性以及用户粉丝的网络结构特性出发,对已有的微博文本 以及时间序列特征进行扩充,提出利用用户粉丝的网络结构向量表示特征,以及 Hashtag 的情感性,地域性,人物性等特征进行模型训练。实验结果表明,新提 出来的特征对于 Hashtag 的流行度预测问题有了一定的效果,也即证明新提出的 特征有效的表达了 Hashtag 的传播特性。

\item \bfseries 基于多源头的主题标签流行度预测 \mdseries 

结合外部知识,利用已有的用户粉丝网络结构作为先验知识,提出多源头的 主题标签流行度预测模型:利用已知的用户粉丝网络结构,学习用户的向量表 达,将其和 Hashtag 的传播路径作为模型的输入进行训练学习,实现用户粉丝网 络结构和多源模型的整合。在大规模微博语料上,对多源模型进行了实验验证。 实验结果表明,模型可以有效地处理多源头的主题标签的流行度预测问题。

\item  \bfseries 建立了一个事件热度预测系统 \mdseries 
事件热度预测的实现与部署,论文考虑到 Hashtag 可以作为微博中的事件或 者话题,采用基于特征的 Hashtag 流行度预测模型,搭建了事件热度预测系统。 事件热度预测系统通过自动分析微博数据,利用本文提出的主题标签流行度预 测算法,进行事件热度预测,通过系统验证了模型的有效性。结果表明,本文提 出的事件热度预测具有较高的应用价值。
\end{enumerate}

\section{下一步工作}

本文的下一步研究内容主要集中在以下几个方面:
\begin{enumerate}
\item  在对主题标签流行度预测中涉及到的特征进行分析后,发现了Hashtag里 面出现的人名以及情感性的重要性,但是在实际的微博消息中,对于更加细致的
 刻画 Hashtag 中的事件性以及不同 Hashtag 之间本质上的区别还是需要更多的研 究工作。
\item 在基于多源头的主题标签流行度预测中,目前考虑了它们最后的融合结 果,但是没有考虑在 Hashtag 传播过程中,多源路径下的用户之间的互相影响, 后续可以试图挖掘不同源头之间的互相影响。
\end{enumerate}